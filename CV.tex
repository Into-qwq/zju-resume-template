\documentclass[11pt]{article}


\setlength{\parindent}{0pt}
\usepackage{xltxtra}
\usepackage{hyperref}
\hypersetup{hidelinks}
\usepackage{url}
\urlstyle{tt}
\usepackage{xcolor}
\definecolor{CVBlue}{RGB}{23,110,191}
\usepackage{calc}
\usepackage{graphicx}
\usepackage{tikz}
\usetikzlibrary{calc}
\usepackage{fontspec}
\usepackage{xeCJK}
\usepackage{enumitem}
\CJKsetecglue{} %% 取消中文与数字之间的间隙


%% 主文档字体设置
\setmainfont[
    Path = fonts/Main/,
    Extension = .otf,
    BoldFont = texgyretermes-bold.otf, % 加粗字体
]{texgyretermes-regular.otf} % 正文字体

% 中文字体设置
\setCJKmainfont[
    Path = fonts/hansans/,
    Extension = .ttf,
    BoldFont = NotoSansSC-Bold.ttf, % 加粗字体
]{NotoSansSC-Regular.otf} % 正文字体


\usepackage{relsize}
\usepackage{xspace}

% 使用 fontawesome(部分图标)
\usepackage{fontawesome} 

% A4纸,上下左右边距
\usepackage[
    a4paper,
    left=1.2cm,
    right=1.2cm,
    top=1.5cm,
    bottom=1cm,
    nohead
]{geometry}

\renewcommand{\baselinestretch}{1.5} % 行间距设为1.5

\usepackage{titlesec}
\usepackage{enumitem}
\setlist{noitemsep} % 取消列表项间的额外间距
%\setlist{nosep} % 取消所有垂直间距
\setlist[itemize]{topsep=0.25em, leftmargin=*}
\setlist[enumerate]{topsep=0.25em, leftmargin=*}

% --- 用于控制【不同项目之间】的垂直距离 ---
\newlength{\interProjectSpacing}
\setlength{\interProjectSpacing}{0.9em} % <--- 在此调整项目之间的距离
\newcommand{\projectsep}{\vspace{\interProjectSpacing}}

% --- 用于控制【项目标题】与下方【项目描述】的距离 ---
\newlength{\intraProjectTitleSep}
\setlength{\intraProjectTitleSep}{0.4em} % <--- 在此调整标题和描述的距离
\newcommand{\titlebreak}{\\[\intraProjectTitleSep]}

% --- 用于控制【项目描述】与下方【要点列表】的距离 ---
\newlength{\intraProjectListTopSep}
\setlength{\intraProjectListTopSep}{0.2em} % <--- 在此调整描述和列表的距离

% =======================================================================


\titleformat{\section}         % 定制 \section 命令 
{\large\bfseries\raggedright} % 将 section 标题设置为大号、粗体且左对齐
{}{0em}                      % 可用于为所有 section 添加前缀(如“章节...”)
{}                           % 可用于在标题前插入代码
[{\color{CVBlue}\titlerule}]  % 在标题后插入一条横线
\titlespacing*{\section}{0cm}{*1.6}{*1.2}



\begin{document}
\pagenumbering{gobble}

%%%% 利用tikz来定位照片
\begin{tikzpicture}[remember picture, overlay] 
    \node[anchor = north east] at ($(current page.north east)+(-2cm,-1.2cm)$) {\includegraphics[height=3cm]{avatar.jpg}};
  \end{tikzpicture}%
  %%%% 利用tikz来定位学校Logo,这里只在第一页显示
  \begin{tikzpicture}[remember picture, overlay] 
    \node[anchor = north west] at ($(current page.north west)+(0.5cm,+1.0cm)$) {\includegraphics[height=6cm]{zju.png}};
  \end{tikzpicture}%
\centerline{\LARGE\bfseries{童熙年}} 

\centerline{\normalsize{\faPhone\ 152-0715-7263 \quad \faEnvelopeO\ \href{mailto:3250105882@zju.edu.cn}{3250105882@zju.edu.cn}}} 

\centerline{\normalsize{\faGithubSquare\ \href{https://github.com/Into-qwq}{https://github.com/Into-qwq} \quad \faRssSquare\ \href{https://www.cnblogs.com/into-qwq}{https://www.cnblogs.com/into-qwq}}} 
    
\section{\makebox[\widthof{\faGraduationCap}][c]{\color{CVBlue}\faGraduationCap}\ 教育背景}    
\textbf{浙江大学} \hfill 2025.9 -- 至今\\[0.5em] % 标题和正文间加一点距离
工科试验班(海洋)\quad 大一 
% \begin{itemize}[nosep]
%     \item 相关课程:《烂坑挖掘及基础》、《高级挖坑技巧》、《烂坑数理统计》
% \end{itemize}

\section{\makebox[\widthof{\faFileText}][c]{\color{CVBlue}\faFileText}\ 学习成绩}

大一上:

\begin{itemize}[nosep, topsep=\intraProjectListTopSep]
    \item 均绩:4.40
    \item 微积分:4.2(86)
    \item 线性代数:4.8(93)
    \item c程:4.8(94)
\end{itemize}

\section{\makebox[\widthof{\faUsers}][c]{\color{CVBlue}\faUsers}\ 项目经历}

% --- 第一个项目 ---
% 将标题行末尾的 \\ 替换为 \titlebreak 命令

\textbf{信息竞赛公开赛} \hfill 2023.01.18 -- 2024.05.01 \titlebreak

比赛名称:\href{https://www.luogu.com.cn/contest/116062}{QWOI Round 1}

举办时间:2024.05.01

组织规模:参赛者3.9k

核心职责与成就:
\begin{itemize}[nosep, topsep=\intraProjectListTopSep]
    \item 组建团队,为比赛分工。
    \item 负责第一第二题的出题和题解,第三第四题的题目验证。
    \item 负责赛时答疑,比赛中未出现重大失误。
    \item 负责赛后讲题,比赛顺利收尾。
\end{itemize}

% 使用 \projectsep 命令来分隔两个项目

\section{\makebox[\widthof{\faCogs}][c]{\color{CVBlue}\faCogs}\ 技术栈}
\begin{itemize}[nosep]
    \item \textbf{编程语言:} \textbf{C++}, Python
    \item \textbf{开发工具:} Vscode, Git, LaTex
    \item \textbf{操作系统:} Windows
\end{itemize}
\section{\makebox[\widthof{\faGraduationCap}][c]{\color{CVBlue}\faList}\ 获奖情况}
\begin{itemize}
    \item 2020	CSP-J	一等奖
    \item 2020	CSP-S	三等奖
    \item 2021	CSP-S	二等奖
    \item 2022	NOIP	二等奖
    \item 2023	CSP-S	一等奖
    \item 2023	NOI 春季测试	一等奖
    \item 2023	NOIP	一等奖
    \item 2023	NOI 冬令营	铜牌
    
\end{itemize}
    
\section{\makebox[\widthof{\faInfo}][c]{\color{CVBlue}\faInfo}\ 其他}
\begin{itemize}[parsep=0.5ex]
    \item \textbf{博客:} \href{https://www.cnblogs.com/into-qwq}{https://www.cnblogs.com/into-qwq}
    \item \textbf{GitHub:} \href{https://github.com/Into-qwq}{https://github.com/Into-qwq} 
\end{itemize}
\end{document}
